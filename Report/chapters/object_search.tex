\chapter{Object-Search Methods}
\label{sec:object_search}
In this chapter different methods for object search and delivery tasks are presented. The agent has to perform a search and delivery task in a closed environment involving multiple items and with prior uncertain knowledge about the items location. First, a more formal problem definition is given in \ref{sec:problemdef} and the underlying assumptions are discussed. In section \ref{sec:baseline} a greedy next-best-view algorithm is described which serves as the baseline for the POMDP agents. Section \ref{sec:POMDPagent} covers the \textcolor{red}{basic} POMDP agent and the underlying system architecture. Finally, in section \ref{sec:MultiScale} a novel MultiScale POMDP solving method is presented. The original POMDP structure is split into multiple layers of POMDPs with different spatial-resolutions. The problem is first solved in the lowest resolution POMDP and its solution is then refined in the lower layers. This solving method yields large speedups while still producing qualitative solutions.  

\section{Problem Definition}\label{sec:problemdef}
\textit{Problem statement:} 
The agent's task is to find one or multiple items in an office environment with multiple rooms and deliver each item to a prior specified goal location in as little time as possible. A map of the static, closed environment is given and there are no furniture or other obstacles present. The items are therefore assumed to lie on the floor. The agent has \textit{prior knowledge} about the item locations in form of a probability distribution available. Their goal location is specified with coordinates. It is further assumed that only one item of each kind is present. Therefore, if the task is to find and deliver a mug, then there is only one mug in the environment. The agent modelled for this task a two-wheeled differential-drive robot shown in figure \ref{fig:robot} with a camera facing in the forward direction of the robot and a robotic arm manipulator mounted on top.\\

At first glance the above assumptions may seem too restricting for a real-world application. However, all methods presented in this chapter can be generalized to broader settings involving dynamic environments, furniture and multiple items of each kind as will be discussed in the respective sections. \textcolor{red}{figure of robot here}
%%%%%%%%%%%%%%%%%%%%%%%%%%%%%%%%%%%%%%%%%%%%%%%%%%%%%%%%%%%%%%%%%%%%%%%%%%%%%%%%%%%%%%%%%%%%%%%%%%%%%%%%%%%%%%%%
%%%%%%%%%%%%%%%%%%%%%%%%%%%%%%%%%%%%%%%%%%%%%%%%%%%%%%%%%%%%%%%%%%%%%%%%%%%%%%%%%%%%%%%%%%%%%%%%%%%%%%%%%%%%%%%%
%%%%%%%%%%%%%%%%%%%%%%%%%%%%%%%%%%%%%%%%%%%%%%%%%%%%%%%%%%%%%%%%%%%%%%%%%%%%%%%%%%%%%%%%%%%%%%%%%%%%%%%%%%%%%%%%
\section{Baseline Agent}\label{sec:baseline}

%%%%%%%%%%%%%%%%%%%%%%%%%%%%%%%%%%%%%%%%%%%%%%%%%%%%%%%%%%%%%%%%%%%%%%%%%%%%%%%%%%%%%%%%%%%%%%%%%%%%%%%%%%%%%%%%
\subsection{Extensions}
%%%%%%%%%%%%%%%%%%%%%%%%%%%%%%%%%%%%%%%%%%%%%%%%%%%%%%%%%%%%%%%%%%%%%%%%%%%%%%%%%%%%%%%%%%%%%%%%%%%%%%%%%%%%%%%%
%%%%%%%%%%%%%%%%%%%%%%%%%%%%%%%%%%%%%%%%%%%%%%%%%%%%%%%%%%%%%%%%%%%%%%%%%%%%%%%%%%%%%%%%%%%%%%%%%%%%%%%%%%%%%%%%
%%%%%%%%%%%%%%%%%%%%%%%%%%%%%%%%%%%%%%%%%%%%%%%%%%%%%%%%%%%%%%%%%%%%%%%%%%%%%%%%%%%%%%%%%%%%%%%%%%%%%%%%%%%%%%%%
\section{POMDP Agent}\label{sec:POMDPagent}
%%%%%%%%%%%%%%%%%%%%%%%%%%%%%%%%%%%%%%%%%%%%%%%%%%%%%%%%%%%%%%%%%%%%%%%%%%%%%%%%%%%%%%%%%%%%%%%%%%%%%%%%%%%%%%%%
\subsection{System Architecture}\label{subsec:sysarch}
\begin{itemize}
    \item receeding horizon solving
    \item figure with 3 level architecture 
    \item synchronicity figure
    \item grid, nodes, belief grid
\end{itemize}
\subsubsection{Grid}
\subsubsection{Nodes}
\subsubsection{Belief Grid}
%%%%%%%%%%%%%%%%%%%%%%%%%%%%%%%%%%%%%%%%%%%%%%%%%%%%%%%%%%%%%%%%%%%%%%%%%%%%%%%%%%%%%%%%%%%%%%%%%%%%%%%%%%%%%%%%
\subsection{POMDP Model}
\textcolor{red}{REFERENCE TO MASTER THESIS/PAPER THIS MODEL IS BASED ON}\\
In this section the POMDP model\footnote{Note that the model presented here is - as is always the case in modelling - a simple approximation of the true system and its complexity is the solely the designers choice.} (see section \ref{sec:POMDP}) defined by the tuple $P = \langle \mathcal{S}, \mathcal{A}, \mathcal{O}, T, Z, R, b_0, \gamma \rangle$ is described for a search task involving $m$ items. The environment is divided into $N$ separate non-overlapping regions which from now on are called nodes. In figure \ref{fig:nodes} a small office environment with 3 rooms is showed which got partitioned into $N=9$ nodes $n_0$ to $n_8$. Note that the nodes can in principle have any shape and do not need to be rectangles.\textcolor{red}{NODES FIGURE HERE}\\
\subsubsection{State Space $\mathcal{S}$}
The state $s$ is represented as a vector of size $m+1$ with one variable describing the agents position and $m$ variables for the items. The agents position $x_a$ is modelled as fully observable and takes the different node numbers as values. Since the area of a node is chosen to take multiple square meters this modeling assumption is well justified. The item variable $x_k$ of item $k$ are partially observable, as their position is not known beforehand. In in addition to the node numbers $x_i$ takes the values \texttt{agent} when the item is on the agent and the value \texttt{goal} after it got delivered to its goal location.

\begin{equation}
    s=\begin{pmatrix} x_a \\ x_1 \\ \vdots \\ x_m \end{pmatrix}, \qquad \begin{aligned} x_a&\in\{n_0, \cdots, n_N\},\\ 
    x_k &\in \{n_0, \cdots, n_N, \texttt{agent}, \texttt{goal}\}, \quad 1 \leq k \leq m \\
    |\mathcal{S}| &= N(N+2)^m \end{aligned}
\end{equation}
Since the object search and delivery task has a defined and observable goal, a set of terminal states can be constructed. A state is a terminal state, iff\footnote{if and only if} all item variables take the value \texttt{goal}
\begin{equation}\label{eq:s_t}
    \mathcal{S}_t = \{s=\begin{pmatrix} x_a & x_1 & \cdots & x_m \end{pmatrix}^T | x_a\in \{n_0, \cdots, n_N\}, x_1=\ldots=x_m=\texttt{goal}\}
\end{equation}
\subsubsection{Action Space $\mathcal{A}$}
The action space consists of different navigation actions that navigate the agent in between nodes, a \texttt{look\_around} action that aims to observe the entire node the agent is currently in, one pickup action for each item and a \texttt{release} action to release the item the agent is carrying. A navigation action \texttt{nav($n_i, n_j$)} takes the agent from node $n_i$ to $n_j$ if he currently is in node $n_i$ and vice versa. Navigation actions are only available between two adjacent nodes. See figure \ref{fig:nodes_graph} where the nodegraph of the three-room environment and the navigation actions are visualized. Other choices for navigation actions that require less actions in total would be possible. An example for such alternative choice is one navigation action per node to navigate the agent to this node. However, the observation model $Z(o, s', a)$ only considers the next state $s'$, not the current state $s$. Clearly, the observation probability of observing an item when choosing a navigation action to node $n_j$ depends on the starting node and the path chosen. With the above choice of navigation action the starting node is implicitly given. Further, by only considering adjacent nodes, the set of possible paths in between the nodes is greatly reduced. This choice of navigation actions therefore allows accurate modelling of the observation function.\\
The \texttt{pickup$k$} action tries to pick item $k$ in the current node the agent is in. Note that a specific pickup action for each item is required to allow the agent to reason which item to pickup if multiple items are in the same node. Since the agent can only carry one item at a time, one release action for all items suffices. 

\begin{equation}
    \mathcal{A} = \{\texttt{nav($n_0, n_1$)}, \cdots, \texttt{look\_around}, \texttt{pickup$1$}, \cdots, \texttt{pickup$m$}, \texttt{release}\}
\end{equation}

\subsubsection{Observation Space $\mathcal{O}$}
The observation $o$ is represented as a vector of size $m$, where each variable corresponds to an item. The observation $o_k$ takes the value \texttt{no} if item $k$ is not observed in this time step, $n_i$ if the item was observed in node $i$ and \texttt{agent} if the agent carries item $k$. If the item variable $x_k$ takes value \texttt{goal} the observation is always \texttt{no}.
\begin{equation}
    o = \begin{pmatrix} o_1 \\ \vdots \\ o_m \end{pmatrix}, \qquad \begin{aligned} o_k &\in \{ \texttt{no}, n_0, \cdots, n_N, \texttt{agent} \}, \quad 1 \leq k \leq m\\
    |\mathcal{O}|&= (N+2)^m \end{aligned}
\end{equation}
\subsubsection{Transition Model $T$}
The transition model $T(s, a, s')$ defines the probability that the agent transitions within one time step to state $s'$ after choosing action $a$ in state $s$. In this model, an entirely deterministic transition model is chosen, i.e. all entries are either one or zero. This implies that given state $s$ and action $a$, the next state $s'$ is perfectly known. Further, conditional independence between the state variables of the next state $s'$ given action $a$ and the current state $s$ is assumed. The transition function can therefore be calculated by multiplying the transition model for each state variable:
\begin{equation}
    \left(x_a' \upmodels x_1' \upmodels \cdots \upmodels x_m' |a, s\right) \Rightarrow T(s,a,s') = T(s,a, x_a') \cdot T(s,a,x_1') \cdot \ldots \cdot T(s,a,x_m').
\end{equation}
This assumption reduces the number of entries in the transition model from \\$\left(N(N+2)\right)^2\cdot|\mathcal{A}|$ to $N(N+2)\cdot|\mathcal{A}|\cdot (N+(N+2)\cdot m)$. The transition function $T(s, a, x_a')$ is given in equations \ref{eq:Tnavxa} - \ref{eq:Tstxa}. Equation \ref{eq:Tnavxa} formalizes that the agent transitions from position $x_a=n_i$ to node $n_j$ and vice versa if the corresponding navigation action \texttt{nav($n_i,n_j$)} is chosen.

\begin{equation}\label{eq:Tnavxa}
    T(s, a=\texttt{nav($n_i, n_j$)}, x_a') = \begin{cases}
             1 & \text{if }x_a=n_i \land x_a'=n_j\\
             1 & \text{if }x_a=n_j \land x_a'=n_i\\
             0 & \text{otherwise}
         \end{cases}
\end{equation}

For the actions \texttt{look\_around}, \texttt{release} and pickup actions the agent remains in the same node: 

\begin{equation}
    T(s,a\in\{\texttt{look\_around}, \texttt{pickup}1,\cdots,\texttt{pickup}m, \texttt{release}\}, x_a') = 
    \begin{cases}
        1 & \text{if }x_a=x_a' \\
        0 & \text{otherwise}
    \end{cases}.
\end{equation}

Finally, if the current state is a terminal state $s_t$, the agent remains per definition in that terminal state

\begin{equation}\label{eq:Tstxa}
    T(s\in\mathcal{S}_t, a, x_a') = \begin{cases}1 &\texttt{if }x_a'=x_a\\
         0& \texttt{otherwise}\end{cases}.
\end{equation}

The transition function $T(s, a, x_k')$ for item $k$ is defined in equations \ref{eq:Txkremain} - \ref{eq:Txkts}. For navigation actions and the \texttt{look\_around} action the items remain in the same state. The navigation action only affects variable $x_a$. If the agent is carrying item $k$, the item takes value \texttt{agent} which is unaffected by navigating. The \texttt{look\_around} action is used to search the current node for items, but does not directly affect the items.    

\begin{equation}\label{eq:Txkremain}
    T(s, a=\in\{\texttt{nav($n_i, n_j$)},\texttt{look\_around}\}, x_k') = \begin{cases}
             1 & \text{if }x_k'=x_k\\
             0 & \text{otherwise}
         \end{cases}
\end{equation}

If the agent chooses the \texttt{pickup} action the item transitions to state \texttt{agent} given that the agent is not carrying an other item and is in the same node as the item currently is. Otherwise the item remains in the same state. Note that a deterministic pickup model is a strong simplification, as the pickup-subroutine (see subsection \ref{subsec:subroutine}) can fail even if $x_a=x_k$ since it's not guaranteed that the agent observes the item. \textcolor{red}{1 SENTENCE WHY THIS SIMPLIFICATION IS OK}

\begin{equation}\label{eq:Txkpickup}
T(s,a=\texttt{pickup}k, x_k') = \begin{cases}
             1 & \text{if }x_a=x_k \land x_i\neq \texttt{agent}, 1\leq i\leq m \land x_k'=\texttt{agent} \\
             1 & \text{if }x_a \neq x_k \land x_k=x_k' \\
             0 & \text{otherwise}
         \end{cases}
\end{equation}

Equation \ref{eq:Txkrelease} shows the transition model for the \texttt{release} action. If the agent is carrying item $k$ and is in the node where the goal location $g_k$ of item $k$ is located, then the item gets successfully delivered within this time step and takes value $\texttt{goal}$. If the release action is chosen in a different node the item is dropped at the agents current location and therefore takes a node value. Item $k$ is not affected by the \texttt{release} action in case the agent is not carrying item $k$.

\begin{equation}\label{eq:Txkrelease}
    T(s,a=\texttt{release}, x_k') = \begin{cases}
             1 & \text{if }x_k =\texttt{agent} \land x_a = g_k \land x_k' = \texttt{goal} \\
             1 & \text{if }x_k =\texttt{agent} \land x_a = x_k' \land x_a \neq g_k \\
             1 & \text{if }x_k\neq \texttt{agent} \land x_k'=x_k \\
             0 & \text{otherwise} 
             \end{cases}
\end{equation}

The item $k$ is not allowed to change value if the current state is a terminal state.

\begin{equation}\label{eq:Txkts}
    T(s\in\mathcal{S}_t, a, x_k') = \begin{cases}1 &\texttt{if }x_k=x_k'\\
         0& \texttt{otherwise}\end{cases}
\end{equation}


\subsubsection{Observation Model $Z$}
The observation model $Z(o, s', a)$ represents the probability of observing observation $o$ given that the time step ends in state $s'$ after choosing action $a$. In this model conditinal independence between the observation variables $o_k$ is assumed, given $s'$ and $a$. The total observation function $Z(o, s', a)$ can therefore be written as the product of the individual observation functions $Z(o_k, s', a)$ which simplifies the design process and reduces the number of values.

\begin{equation}\label{eq:Ocondind}
    \left(o_1 \upmodels \cdots \upmodels o_m |s', a\right) \Rightarrow Z(o,s',a) = \prod_{1\leq k \leq m}Z(o_k,s',a)
\end{equation}

Equation \ref{eq:Onav} - \ref{eq:Orelease} show the observation function for item $k$ for the different actions the agent can choose. The probability of observing the item during a navigation action in node $n_u$ is approximated as the ratio between the expected observed area of node $n_u$ and the total area of node $n_u$. See figure \textcolor{red}{FIG REF HERE} for a visualization of the expected observed area. The ratio gets multiplied by the constant "object recognition rate", abbreviated as $orr$, which takes into account that the object recognition algorithm could fail to correctly recognize an item even if it is present. Note that this model does not consider which part of the area of the node is observed. If the agent were to execute the same navigation action a second time with the same starting end ending node, the observed area will be the same as in the first navigation action. The agent however would still expect to make new observations. This model inaccuracy could in the worst case lead to the agent driving back and forth between the same nodes in an attempt to observe the nodes. Since the algorithm is run in a receeding horizon fashion, this inaccuracy can be corrected for by setting the observation probability for a navigation action to zero if it was chosen in previous time steps.\\
The probability of not observing the item even tough it is present in node $n_u$ is given by one minus the probability of observing the item. If the agent carries the item after the action finished, the observation is always $\texttt{agent}$ independent of which action was chosen. Since an item that already got delivered must not longer be considered the observation of that item is assumed to be \texttt{no}. All other cases like that the object recognition algorithm wrongly observes item $k$ in a node where it is not present are neglected and set to zero probability. 

\begin{equation}\label{eq:Onav}
    Z(o_k, s', a=\texttt{nav($n_i, n_j$)}) = \begin{cases} orr \cdot \frac{\text{observed area of }n_u}{\text{area of }n_u} & \text{if }o_k=x_k'=n_u \\ 1 - orr \cdot \frac{\text{observed area of }n_u}{\text{area of }n_u} & \text{if } o_k=\texttt{no}\land x_k'=n_u\\
    1 & \text{if }o_k=x_k'=\texttt{agent}\\
    1 & \text{if }o_k=\texttt{no}\land x_k'=\texttt{goal}\\
    0 & \text{otherwise}
    \end{cases}
\end{equation}

For the \texttt{look\_around} action the probability of observing the item in the current node is chosen as ninety percent. Consequently the probability of not observing the item is set to ten percent. The fact that the item could be observed in a different node than the one the agent is currently examining is neglected. 

\begin{equation}\label{eq:lookaround}
    Z(o_k, s', a=\texttt{look\_around}) = \begin{cases}
    0.9& \text{if } o_k=x_a'=x_k'\\
    0.1 &\text{if } o_k=\texttt{no} \land x_a'=x_k'\\
    1& \text{if }o_k=x_k'=\texttt{agent}\\
    1& \text{if }o_k=\texttt{no}\land x_k'=\texttt{goal}
    \\0 &\text{otherwise} \end{cases}
\end{equation}

A pickup action is assumed to always succeed if the agent observes the item in the current node and to fail otherwise. The probability of observing \texttt{no} is therefore one if the agent is not carrying the item after the action is finished. If the action succeeds the item is on the agent and the observation is \texttt{agent}. 

\begin{equation}\label{eq:pickup}
    Z(o_k, s', a=\texttt{pickup}k) = \begin{cases}
    1 &\text{if } o_k=\texttt{no} \land x_k'\neq \texttt{agent}\\
    1& \text{if }o_k=x_k'=\texttt{agent}\\
    1& \text{if }o_k=\texttt{no}\land x_k'=\texttt{goal}
    \\0 &\text{otherwise} \end{cases}
\end{equation}

If the agent carries item $k$ and chooses the release action the next item state is already known. No informative observation is needed. Therefore the probability of observing \texttt{no} is set to one. The probability of observing a different item during releasing item $k$ is neglected. Since releasing an item for the sake of observing a different item is most likely not a time-efficient behaviour this simplification is well justified. Similarly, the probability of observing item $k$ during the pickup action of item $i$ where $k\neq i$ is assumed to be zero. 

\begin{equation}\label{eq:Orelease}
    Z(o_k, s', a\in\{\text{pickup}i, \texttt{release}\}) = \begin{cases}
    1& \text{if } o_k=\texttt{no}\\
    0& \text{otherwise}\end{cases}
\end{equation}


\subsubsection{Reward Model $R$}
Reward models are used to capture the goals of the task the agent should solve. The goal of this object search and delivery problem is to find the items and deliver them to their target locations in as little time as possible. The reward function used for this task can be split into a time penalty $R_\text{time}$, a reward  $R_\text{subtask}$ for picking an item up and a reward $R_\text{task}$ for delivering an item to its target location. The time penalty makes sure that the optimal policy of the POMDP minimizes the expected completion time of the deliveries, while the subtask and task reward guide the solver towards the right solution. 

\begin{equation}
    R = R_{\text{time}} + R_{\text{subtask}} + R_{\text{task}} 
\end{equation}
Execution times of the actions in the POMDP differ and are not previously known, since they are high-level actions that are processed by their designated subroutine. The run-time of the subroutines depend on the specific agent location, item positions and the current state of the belief grid. Modelling $R\text{time}$ requires approximating these execution times, either through a simulation using sampling to get robust averaged values or by using a heuristic. In the implementation of this model the simulation approach was used.\\
The expected run-time of the navigation function is given by $t_\texttt{nav}(n_i, n_j)$ where $n_i$ is the start node and $n_j$ the target node. To avoid that the agent picks a navigation action that is not available in the current agents position, a high negative reward is given for 
\begin{equation}
    R_\text{time}(s, a=\texttt{nav($n_i, n_j$)}) = \begin{cases}
    -t_\texttt{nav}(n_i, n_j) & \text{if } x_a=n_i\\
    -t_\texttt{nav}(n_j, n_i) & \text{if } x_a=n_j\\
    -1000 & \text{otherwise}
    \end{cases}
\end{equation}
The expected time $t_\texttt{l.a}$ of the \texttt{look\_around} action is node-dependent since nodes usually have different shapes and sizes. Further, $t_\texttt{l.a}$ differs dependent on if an item is present or not as is explained in subsection \ref{subsec:subroutine}. 

\begin{multline}
    R_\text{time}(s, a=\texttt{look\_around}) = \\\begin{cases}
    -t_\texttt{l.a}(n_i, \texttt{item\_present=True}) & \text{if }x_a=n_i \land x_k=n_i, \quad 1\leq k \leq m\\
    -t_\texttt{l.a}(n_i, \texttt{item\_present=False}) & \text{if }x_a=n_i \land x_k \neq n_i,\quad  1\leq k \leq m
    \end{cases}
\end{multline}

For the pickup action  the same considerations as for the \texttt{look\_around} action are made. The time penalty $R_\text{time}$ is given by
\begin{multline}
    R_\text{time}(s, a=\texttt{pickup}k) = \\ \begin{cases}
    -t_\texttt{pickup}(n_i, \texttt{item\_present=True}) & \text{if }x_a=n_i \land x_k=n_i, \quad 1\leq k \leq m\\
    -t_\texttt{pickup}(n_i, \texttt{item\_present=False}) & \text{if }x_a=n_i \land x_k \neq n_i,\quad  1\leq k \leq m
    \end{cases}.
\end{multline}

\textcolor{red}{IF TIME: SPECIFY HOW THE EXPECTED TIME FOR THE PICKUP ACTION IS CALCULATED}
% Note that the choice of $t_\texttt{pickup}(n_i, \texttt{item\_present=False})$ can be critical for a good performance. Ideally, the agent chooses the \texttt{look\_around} action if it isn't sure if the item is present. If the item is then observed a pickup action with expected time $t_\texttt{pickup}(n_i, \texttt{item\_present=True})$ should be chosen. If $t_\texttt{pickup}(n_i, \texttt{item\_present=False})$ is chosen too low, the agent will always choose the pickup action instead of the \texttt{look\_around} action, which yields bad performance as the pickup subroutine is not made for exploration. 


% \begin{multline}
% t_\texttt{pickup}(n_i, \texttt{item\_present=False}) > \\
%     t_\texttt{l.a}(n_i, \texttt{item\_present=False}) + t_\texttt{l.a}(n_i, \texttt{item\_present=True})
% \end{multline}

Finally, for the release action the expected execution time differs dependent upon if agent releases item $k$ in the goal node or not. In the goal node the agent first has to drive to the goal coordinates, while in any other node the item is dropped at the current agents location. 
\begin{equation}
     R_\text{time}(s, a=\texttt{release}) = \begin{cases}
    -t_\texttt{release\_goal}(n_i) & \text{if }x_a=g_k \land x_k=\texttt{agent}, \quad 1 \leq k \leq m \\
    -t_\texttt{release\_drop}(n_i) & \text{if }x_a\neq g_k \land x_k=\texttt{agent}, \quad 1 \leq k \leq m\\
    0 & \text{if }x_k \neq \texttt{agent}, \quad 1 \leq k \leq m
    \end{cases}
\end{equation}

The agent receives a positive reward $r_\text{pickup}$ for successfully picking an item up. When the agent releases the item in a later time step the same reward must be subtracted. Without the negative reward for releasing an item, the agent would choose to pick an item up and release it right afterwards in an infinite loop to gather the subtask reward. Note that $R_\text{subtask}(s,a,s')$ is written as a function of the current state $s$, action $a$ and the next state $s'$. Indeed, all the equations in chapter \ref{chap:probabilistic} can be adapted without further adjustments to a reward model that is a function of $s'$ as well. 
 
\begin{equation}
    R_\text{subtask}(s,a,s') = \begin{cases}
    +r_\text{pickup} & \text{if }a=\texttt{pickup}k \land x_k'=\texttt{agent}\\
    -r_\text{pickup} & \text{if }a=\texttt{release}\land x_k=\texttt{agent}\\
    0 & \text{otherwise}\end{cases}
\end{equation}

Similarly, there is a positive reward $r_\text{delivery}$ for delivering an item to its goal location. The transition model does not allow the agent to pick an item up if it takes the value \texttt{goal}, therefore no negative reward for this case is required. 

\begin{equation}
    R_\text{task}(s,a,s') = \begin{cases} + r_\text{delivery} & \text{if }a=\texttt{release}\land x_k'=\texttt{goal}\land x_k=\texttt{agent}\\
    0 & \text{otherwise}
    \end{cases}
\end{equation}


\subsubsection{Initial Belief $b_0$}
The initial belief $b_0(s)$ is a distribution that defines the probability for each value of the partially observable variables in state $s$. Since the belief grid (see subsection \ref{subsec:sysarch}) considers the items independent of each other, the belief can be expressed as multiplication of the belief $b_0(x_k)$ of each item
\begin{equation}
    b_0(s) = \prod_{1\leq k \leq m} b_0(x_k).
\end{equation}

The belief $b_0(x_k)$ of item $k$ can be represented as a vector where each element corresponds to a probability of an item value 
\begin{equation}
    b_0(x_k) = \begin{pmatrix} b_0(x_k=n_0) & \cdots & b_0(x_k=n_N) & b_0(x_k=\texttt{agent}) & b_0(x_k=\texttt{goal}) \end{pmatrix}^\intercal.
\end{equation}

To calculate the probability that the item is in node $n_i$, the cells $\mathcal{C}(n_i)$ of the belief grid that lie within $n_i$ are aggregated

\begin{equation}
b_0\left(x_k=n_i\right) = \sum_{j\in \mathcal{C}(n_i)}\texttt{belief\_grid[$j$]}.
\end{equation}

Note that item value \texttt{agent} and \texttt{goal} are observable values and the initial belief is therefore either one if the item is on the agent or already got delivered and zero otherwise. 

\subsubsection{Discount Factor $\gamma$}
Discounting can be used to value future rewards less than current ones and is required for some solving methods like Sarsop (see subsection \ref{subsec:SARSOP}). Since the goal of this object search and delivery problem is to minimize the total time, the discounting factor is chosen close to one:
\begin{equation}
    \gamma = 0.999
\end{equation}

%%%%%%%%%%%%%%%%%%%%%%%%%%%%%%%%%%%%%%%%%%%%%%%%%%%%%%%%%%%%%%%%%%%%%%%%%%%%%%%%%%%%%%%%%%%%%%%%%%%%%%%%%%%%%%%%
\subsection{Subroutines}\label{subsec:subroutine}
\begin{itemize}
    \item \texttt{Nav($n_i$, $n_j$)}
    \item \texttt{pickup}
    \item \texttt{look\_around}
    \item \texttt{release}
\end{itemize}
%%%%%%%%%%%%%%%%%%%%%%%%%%%%%%%%%%%%%%%%%%%%%%%%%%%%%%%%%%%%%%%%%%%%%%%%%%%%%%%%%%%%%%%%%%%%%%%%%%%%%%%%%%%%%%%%
\subsection{Assumptions and Extensions}
\begin{itemize}
    \item Dynamic Environment
    \item Furniture
    \item Multiple items of same kind
\end{itemize}
%%%%%%%%%%%%%%%%%%%%%%%%%%%%%%%%%%%%%%%%%%%%%%%%%%%%%%%%%%%%%%%%%%%%%%%%%%%%%%%%%%%%%%%%%%%%%%%%%%%%%%%%%%%%%%%%
%%%%%%%%%%%%%%%%%%%%%%%%%%%%%%%%%%%%%%%%%%%%%%%%%%%%%%%%%%%%%%%%%%%%%%%%%%%%%%%%%%%%%%%%%%%%%%%%%%%%%%%%%%%%%%%%
%%%%%%%%%%%%%%%%%%%%%%%%%%%%%%%%%%%%%%%%%%%%%%%%%%%%%%%%%%%%%%%%%%%%%%%%%%%%%%%%%%%%%%%%%%%%%%%%%%%%%%%%%%%%%%%%
\section{Multi-Scale POMDP Agents}\label{sec:MultiScale}
%%%%%%%%%%%%%%%%%%%%%%%%%%%%%%%%%%%%%%%%%%%%%%%%%%%%%%%%%%%%%%%%%%%%%%%%%%%%%%%%%%%%%%%%%%%%%%%%%%%%%%%%%%%%%%%%
\subsection{Overview}
%%%%%%%%%%%%%%%%%%%%%%%%%%%%%%%%%%%%%%%%%%%%%%%%%%%%%%%%%%%%%%%%%%%%%%%%%%%%%%%%%%%%%%%%%%%%%%%%%%%%%%%%%%%%%%%%
\subsection{Method 01}
%%%%%%%%%%%%%%%%%%%%%%%%%%%%%%%%%%%%%%%%%%%%%%%%%%%%%%%%%%%%%%%%%%%%%%%%%%%%%%%%%%%%%%%%%%%%%%%%%%%%%%%%%%%%%%%%
\subsection{Method 02}
%%%%%%%%%%%%%%%%%%%%%%%%%%%%%%%%%%%%%%%%%%%%%%%%%%%%%%%%%%%%%%%%%%%%%%%%%%%%%%%%%%%%%%%%%%%%%%%%%%%%%%%%%%%%%%%%
\subsection{Method 03}
%%%%%%%%%%%%%%%%%%%%%%%%%%%%%%%%%%%%%%%%%%%%%%%%%%%%%%%%%%%%%%%%%%%%%%%%%%%%%%%%%%%%%%%%%%%%%%%%%%%%%%%%%%%%%%%%
\subsection{Mixture Methods}
%%%%%%%%%%%%%%%%%%%%%%%%%%%%%%%%%%%%%%%%%%%%%%%%%%%%%%%%%%%%%%%%%%%%%%%%%%%%%%%%%%%%%%%%%%%%%%%%%%%%%%%%%%%%%%%%
\subsection{Properties}
\subsubsection{Suboptimality}
\subsubsection{Stability Issues}

