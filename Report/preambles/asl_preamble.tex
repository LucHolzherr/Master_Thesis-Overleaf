% Encoding settings
\usepackage[utf8]{inputenc}
\usepackage[OT1]{fontenc}

% Paper size
\usepackage{a4}

% Headings
\usepackage{fancyhdr}

% More symbols
\usepackage{textcomp}\usepackage{gensymb}

% Math support for Times font
%\usepackage{txfonts}

% ISO date
\usepackage[english]{isodate}

% Multi column
\usepackage{multicol}

% Figures
\usepackage{graphicx}

% Subfigures (obsolete)
\usepackage{caption}
\usepackage{subcaption}
%\usepackage{subfigure}

% Bibliography
\usepackage[numbers]{natbib}

% Nicer tables
\usepackage{booktabs}
\usepackage{array}
\usepackage{multirow}

% Colors
\usepackage{color}
\usepackage{colortbl}
\definecolor{black}{rgb}{0,0,0}
\definecolor{white}{rgb}{1,1,1}
\definecolor{darkred}{rgb}{0.5,0,0}
\definecolor{darkgreen}{rgb}{0,0.5,0}
\definecolor{darkblue}{rgb}{0,0,0.5}

% Additional math functionality
\usepackage{amsmath}
\usepackage{amssymb}
\usepackage{MnSymbol}
\def\ci{\perp\!\!\!\perp}
\DeclareMathOperator*{\argmax}{arg\,max}
\DeclareMathOperator*{\argmin}{arg\,min}
\DeclareMathOperator{\atantwo}{atan2}

% Nice fractions
\usepackage{nicefrac}

% Upper case greek letters
\usepackage{upgreek}

% ISO math notation
\usepackage{isomath}
\renewcommand{\vec}{\vectorsym}
\newcommand{\mat}{\matrixsym}

% Units
\usepackage{units}
\usepackage{siunitx}

% Rotated objects
\usepackage{rotating}

% Indent
\setlength{\parindent}{0em}

% Include PDF pages
\usepackage{pdfpages}
\includepdfset{pages={-}, frame=true, pagecommand={\thispagestyle{fancy}}}

% Headings
\rhead[\thepage]{\nouppercase{\rightmark}}
\lhead[\nouppercase{\leftmark}]{\thepage}
\cfoot{}

% Gantt chart
\usepackage{pgfgantt}

% Links (last package)
\PassOptionsToPackage{hyphens}{url}\usepackage{hyperref}

% Clever references (has to be loaded after hyperref)
\usepackage{cleveref}

\usepackage{enumerate}  

% self made example environment
% \usepackage{amsthm} 
% \newtheoremstyle{example}{}{}{}{}{\bfseries}{\smallskip}{\newline}{}
% \theoremstyle{example}
% \newtheorem{example}{Example}
% \numberwithin{example}{chapter}

\usepackage{ntheorem}
%\theoremstyle{break}
\theoremstyle{plain}
\theorembodyfont{\upshape}
\newtheorem{example}{Example}
\numberwithin{example}{chapter}

% self made symbol for denoting the end of an example
\newcommand\xqed[1]{%
  \leavevmode\unskip\penalty9999 \hbox{}\nobreak\hfill
  \quad\hbox{#1}}
\newcommand\demo{\xqed{$\square$}}

% self made command
%\newcommand{\SubNode}{\mathit{Subnodes}}
\newcommand{\SubNode}{\mathcal{N}}

% Algorithms
\usepackage[boxed, ruled,]{algorithm2e}
%\newcommand\mycommfont[1]{\footnotesize\ttfamily\textcolor{blue}{#1}}
\newcommand\mycommfont[1]{\footnotesize\ttfamily\textcolor{darkgreen}{#1}}
\SetCommentSty{mycommfont}

\usepackage{hyperref}

% added by me for repeated captions of figure
\newcommand{\repeatcaption}[2]{%
  \renewcommand{\thefigure}{\ref{#1}}%
  \captionsetup{list=no}%
  \caption{#2 (continued from page \pageref{#1})}%
}

\usepackage{float}